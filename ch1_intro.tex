With the increasing frequency of natural language interfaces for consumer
products, natural language generation (NLG) is becoming more and more important
in industry.  Consequently, a consistent and reliable method for creating a natural
language generation system would be of value, especially if the system's
computational requirements were small enough that the system could be embedded
in consumer electronics.  In this work, we propose a system based in statistical
planning which serves as a general-purpose natural language generation system.
As a part of this work, we propose a method for specifying a grammar based on
tree-adjoining grammars using a modified UCT algorithm.

\section{Natural Language Generation Systems}
``Give me a sentence about a cat and a dog.'' Given such a {\em
  communicative goal}, most native English speakers 
can answer a question like this one quite easily.
They can also usually provide several similar sentences,
differing in details but all satisfying the general communicative goal,
and they can usually do this with very little error. Natural language generation (NLG) develops
techniques to extend similar capabilities to automated systems. 
Here, we consider the following restricted NLG problem: given a
grammar, lexicon, and a communicative goal, output a valid English
sentence that satisfies this goal.