Artificial Intelligence (A.I.) systems are becoming increasingly prevalent in
modern consumer products.  Google's ``Google Now" system determines
what information its user wants to see before they ask for it.  Apple's ``Siri"
acts as an artificial personal assistant, attempting to respond to queries stated in
natural language.  Android (which includes Google Now) and iOS (which
includes Siri) have 85\% market penetration between them, and there are
over one billion smartphones active.\\

Visions of future A.I. systems have long included natural
language interfaces.  The ship's computer from Star Trek, the robots from
the works of Heinlein and Asimov, and the droids from Star Wars are
all capable of human speech, and even those which are not capable of dialogue
are able to receive orders verbally and respond in kind.  The creation
of an artificial intelligence like humans have been imagining for many years
requires as a precondition the creation of a system for understanding and
creating language.  It is the latter part of this requirement which we attempt to 
address in this work.\\

Consequently, a crucial subcomponent of artificial intelligence is Natural Language Processing (NLP).
In order to successfully communicate with people, a system which does natural
language processing (the understanding, analysis and use of language) will need to
take in language and translate it into a format that computers can work with.  It will also
need to be able to translate from an internal meaning representation to natural language.
This latter ability is known as Natural Language Generation.\\

Even without the aforementioned lofty future goals, the increasing frequency of natural language interfaces for consumer
products means that natural language generation (NLG) is becoming more and more important
in the world.  Consequently, a consistent and reliable method for creating a natural
language generation system would be of value, especially if the system's
computational requirements were small enough that the system could be embedded
in consumer electronics.  The main contribution of this thesis is a system based in statistical
planning which serves as a general-purpose natural language generation system.\\

In this thesis, we consider the following restricted NLG problem: given a
grammar, lexicon, and a communicative goal, output a valid English
sentence that satisfies this goal.\\

We propose an algorithm which is based in Monte-Carlo Tree Search, an increasingly
popular method for probabilistic planning.  Our algorithm specifically is based in
UCT, the "Upper Confidence bound applied to Trees" algorithm.  This algorithm gives
us the ability to efficiently search a large space (all possible natural language outputs) for 
one of many valid outputs.  We compare our work to that of previous Natural Language
Generation systems, including one which uses traditional AI Planning techniques.
We find that our approach compares favorably in many areas with these state-of-the-art
NLG systems.\\