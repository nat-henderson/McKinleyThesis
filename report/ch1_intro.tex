Visions of future artificial intelligence systems have long included natural
language interfaces.  The ship's computer from Star Trek, the robots from
the works of Heinlein, Asimov, and Dick, and the droids from Star Wars are
all capable of human speech, and even those which are not capable of dialogue
are able to receive orders verbally and respond in kind.  The creation
of an artificial intelligence like humans have been imagining for many years
requires as a precondition the creation of a system for understanding and
creating language.  It is the latter part of this requirement which we attempt to 
address in this work.

\section{Natural Language Generation Systems}
Even without these lofty future goals, the increasing frequency of natural language interfaces for consumer
products means that natural language generation (NLG) is becoming more and more important
in industry.  Consequently, a consistent and reliable method for creating a natural
language generation system would be of value, especially if the system's
computational requirements were small enough that the system could be embedded
in consumer electronics.  In this work, we propose a system based in statistical
planning which serves as a general-purpose natural language generation system.
As a part of this work, we propose a method for specifying a grammar based on
tree-adjoining grammars using a modified UCT algorithm.

``Give me a sentence about a cat and a dog.'' Given such a {\em
communicative goal}, most native English speakers
can answer a question like this one quite easily.
They can also usually provide several similar sentences,
differing in details but all satisfying the general communicative goal,
and they can usually do this with very little error. Natural language generation (NLG) develops
techniques to extend similar capabilities to automated systems. 
Here, we consider the following restricted NLG problem: given a
grammar, lexicon, and a communicative goal, output a valid English
sentence that satisfies this goal.

\section{Contribution}
We propose an algorithm which is based in Monte-Carlo Tree Search, an increasingly
popular method for probabilistic planning.  Our algorithm specifically is based in
UCT, the "Upper Confidence bound applied to Trees" algorithm.  This algorithm gives
us the ability to efficiently search a large space (all possible natural language outputs) for 
one of many valid outputs.  We compare our work to that of previous Natural Language
Generation systems, including one which uses traditional AI Planning techniques.
We find that our approach compares favorably in many areas with these state-of-the-art
NLG systems.