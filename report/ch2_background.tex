\section{Tree Adjoining Grammars}

TAGs are tree-based grammars consisting of two sets of trees, called initial
trees and auxiliary or adjoining trees.  These two kinds of trees generally perform
different roles semantically in addition to their differing syntactic roles.  The former,
initial trees, are usually for adding new semantic information to the sentence.  They
add new nodes to the sentence tree.  In a simplified Context Free Grammar of English,
initial trees contain rules like "Verb Phrases contain a Verb and a Noun", or "VP -> V N".
A sentence can be made entirely of initial trees, but a sentence must contain at least
one initial tree.  An example of an initial tree is shown in Figure <SOMETHING>.\\
(S (NP (D) (N (Cat))) (VP))\\
This tree has as its root the S node, and this defines how it can interact with other
trees under a TAG.  Since this is an initial tree, it can only interact with other trees by
substitution.  That is, this tree is a drop-in replacement for an S node with no children.
This is how we get from our stub sentence (S) to a complete sentence.

Adjoining trees usually clarify a point in a sentence.  In a simplified CFG of English, adjoining
trees would contain rules like "a noun can have an adjective placed in front of it," or "N -> A N".
An example of an adjoining tree is shown in Figure <SOMETHING>.\\
(N (A (Red)) (N*))\\
This tree has as its root an N node.  It also has a specially annotated N node elsewhere in the
tree.  These nodes define its interaction with other trees under a TAG.  Adjoining trees interact
with other trees only by "adjoining".  In an adjoining action, you select the node
to adjoin to, which must be of the same label as the root node of the adjoining tree.  You remove
that node from the other tree and put the adjoining tree in its place.  Then
you place that original node into the adjoining tree as a substitution for the foot node.
For example, if we had the trees\\
(S (NP (D (the)) (N (cat))) (VP))\\
and we wanted to adjoin the example adjoining tree above, we would first create this intermediate tree:\\
(S (NP (D (the)) (N (A (red) (N*)))) (VP))\\
And then perform the substitution:\\
(S (NP (D (the)) (N (A (red) (N (cat))))) (VP))\\
Notice that this has the effect, in all cases, of making the tree deeper.

We use a variation of TAGs in our work, called a lexicalized TAG (LTAG), where each tree is
associated with a lexical item called an anchor.  All examples given above are examples of
lexicalized trees.  An example of an unlexicalized tree would be (NP (D) (N)), where there
are no nodes containing lexical tokens.

\section{Grammar Specification}

A grammar, for our purposes, contains a set of trees, divided into two sets (initial and auxiliary).
These trees need to be annotated with the entities in them.  Entities are defined as any element
anchored by precisely one node in the tree which can appear in a proposition representing the
semantic content of the tree.  These trees are uniquely named, and also contain an
annotation (+) representing the lexicalized node.

\section{Lexicon Specification}

The lexicon that we use is a list of permissible word-tree pairings, annotated with the meaning of
the pairing.  For instance, if the grammar contained a tree named "a.adj", (N (A+) (N-foot*))
(an adjoining tree for prepending an adjective to a noun, whose foot node is an entity named "foot"),
then the lexicon might contain an entry "a.adj: ['red', 'red(foot)]".  That would mean that the overall
LTAG that we are using contains the tree named "a.adj", lexicalized to 'red', and that when that tree
is applied to a sentence, the sentence's meaning is adjusted to include red(name of the foot node).

\section{Communicative Goal Specification}

The communicative goal is just a list of propositions with dummy entity names.  Matching entity
names refer to the same entity; for instance, a communicative goal of 'red(d), dog(d)' would
match a sentence with the semantic representation 'red(subj), dog(subj), cat(obj), chased(subj, obj)',
like "The red dog chased the cat", for instance.  As you can see, the communicative goal does
not have to refer to any "central meaning" of the sentence as a human would select it, but rather
to propositions which the sentence affirms.