In this chapter, we provide some background on natural language generation,
grammars which we considered for use in this work, followed by a discussion
of approaches to Natural Language Generation.

\section{Natural Language Grammars}

Natural languages like English are well-known to have grammars which are difficult
to represent with any single given formalism.  Many attempts have been made to
create a grammar which can sufficiently define natural language, the XTAG project
chiefly among them, but even those projects fail to sufficiently embody the
constraints that spoken and written English put on word orderings and meaning.

\subsection{Tree Adjoining Grammars}

TAGs are tree-based grammars consisting of two sets of trees, called initial
trees and auxiliary or adjoining trees.  These two kinds of trees generally perform
different roles semantically in addition to their differing syntactic roles.  The former,
initial trees, are usually for adding new semantic information to the sentence.  They
add new nodes to the sentence tree.  In a simplified Context Free Grammar of English,
initial trees contain rules like "Verb Phrases contain a Verb and a Noun", or "VP -> V N".
A sentence can be made entirely of initial trees, but a sentence must contain at least
one initial tree.  An example of an initial tree is shown in Figure <SOMETHING>.\\
(S (NP (D) (N (Cat))) (VP))\\
This tree has as its root the S node, and this defines how it can interact with other
trees under a TAG.  Since this is an initial tree, it can only interact with other trees by
substitution.  That is, this tree is a drop-in replacement for an S node with no children.
This is how we get from our stub sentence (S) to a complete sentence.

Adjoining trees usually clarify a point in a sentence.  In a simplified CFG of English, adjoining
trees would contain rules like "a noun can have an adjective placed in front of it," or "N -> A N".
An example of an adjoining tree is shown in Figure <SOMETHING>.\\
(N (A (Red)) (N*))\\
This tree has as its root an N node.  It also has a specially annotated N node elsewhere in the
tree.  These nodes define its interaction with other trees under a TAG.  Adjoining trees interact
with other trees only by "adjoining".  In an adjoining action, you select the node
to adjoin to, which must be of the same label as the root node of the adjoining tree.  You remove
that node from the other tree and put the adjoining tree in its place.  Then
you place that original node into the adjoining tree as a substitution for the foot node.
For example, if we had the trees\\
(S (NP (D (the)) (N (cat))) (VP))\\
and we wanted to adjoin the example adjoining tree above, we would first create this intermediate tree:\\
(S (NP (D (the)) (N (A (red) (N*)))) (VP))\\
And then perform the substitution:\\
(S (NP (D (the)) (N (A (red) (N (cat))))) (VP))\\
Notice that this has the effect, in all cases, of making the tree deeper.

We use a variation of TAGs in our work, called a lexicalized TAG (LTAG), where each tree is
associated with a lexical item called an anchor.  All examples given above are examples of
lexicalized trees.  An example of an unlexicalized tree would be (NP (D) (N)), where there
are no nodes containing lexical tokens.